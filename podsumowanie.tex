\chapter{Podsumowanie}
\thispagestyle{chapterBeginStyle}

Celem niniejszej pracy było opracowanie równoległej wersji algorytmu ewolucyjnego, który rozwiązywałby zadanie transportowe z nieliniową funkcją 
kosztu, oraz przetestowanie go i porównanie z innymi algorytmami służącymi do optymalizacji tego typu zadań. Cel został osiągnięty, a analiza 
eksperymentalna pokazała, że zaimplementowany algorytm osiąga wyniki lepsze, lub bardzo podobne do tych otrzymywanych przy użyciu innych 
metod optymalizacji. Udało się to osiągnąć poprzez dostosowanie reprezentacji chromosomu, jak i poszczególnych operatorów w postaci mutacji 
i krzyżowania do specyfiki zadania transportowego. Dzięki temu użycie operatorów pozwalało na zachowanie ograniczeń zadania. Algorytm działa 
poprawnie nie zależnie od wybranej funkcji kosztu, w dodatku nie wymaga on tego, żeby używana funkcja była różniczkowalna, co jest wymogiem 
w przypadku używania innych dostępnych solverów.

Głównym atutem przedstawionego algorytmu jest szybkość z jaką znajdował rozwiązania, co udało 
się osiągnąć dzięki wprowadzonej równoległości i wykorzystaniu języka Julia, który jest tworzony z myślą o szybkich obliczeniach. Zaprezentowany 
algorytm ewolucyjny potrzebował jedynie kilkudziesięciu sekund na rozwiązanie dużych zadań, których rozmiar wynosił $100 \times 100$, gdzie reszta 
solverów potrzebowała od kilkudziesięciu minut do nawet ośmiu godzin.

% W podsumowanie należy określić stan zakończonych prac projektowych i implementacyjnych. Zaznaczyć, które z zakładanych funkcjonalności systemu udało się zrealizować. Omówić aspekty pielęgnacji systemu w środowisku wdrożeniowym. Wskazać dalsze możliwe kierunki rozwoju systemu, np.\ dodawanie nowych komponentów realizujących nowe funkcje.

% W podsumowaniu należy podkreślić nowatorskie rozwiązania zastosowane w projekcie i implementacji (niebanalne algorytmy, nowe technologie, itp.).



