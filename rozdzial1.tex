\chapter{Analiza problemu}
\thispagestyle{chapterBeginStyle}
\label{rozdzial1}

\section{Zadanie transportowe}
Zadanie transportowe możemy zaliczyć do grupy zadań optymalizacyjnych z ograniczeniami. Rozwiązując je, staramy się przy użyciu $n$ punktów nadania 
zaspokoić zapotrzebowanie $m$ punktów odbioru w taki sposób, aby całkowity koszt transportu był minimalny. Zadanie wymaga określenia ilości 
towaru znajdującego się w każdym z punktów nadania, a także zapotrzebowania na ten towar w każdym z punktów odbioru. Dodatkowo musimy określić 
koszt transportu pomiędzy poszczególnymi punktami. Klasyczne zadanie transportowe ogranicza się do transportu 
tylko jednego rodzaju towaru, dzięki czemu punkty odbioru mogą być zaopatrywane przez jeden lub więcej punktów nadania.

Zadanie transportowe nazywamy zbilansowanym, jeśli całkowita podaż towaru jest równa całkowitemu popytowi. W przeciwnym wypadku 
zadanie jest niezbilansowane. Rozwiązywanie zadania niezbilansowanego polega na sprowadzeniu go do zadania zbilansowanego, poprzez 
dodanie fikcyjnego dostawcy (w przypadku większego popytu), lub fikcyjnego odbiorcy (w przypadku większej podaży). Koszt transportu 
między fikcyjnym dostawcą a odbiorcami, lub między dostawcami a fikcyjnym odbiorcą najczęściej ustalany jest jako zerowy.

Załóżmy, że mamy $n$ punktów nadania i $m$ punktów odbioru. Początkowa ilość towaru w $i$-tym punkcie nadania jest równa $supply(i)$, 
a początkowe zapotrzebowanie w $j$-tym punkcie odbioru jest równe $demand(j)$. Jeśli $x_{ij}$ jest ilością towaru dostarczanego przez 
$i$-ty punkt nadania do $j$-tego punktu odbioru, to zbilansowane zadanie transportowe możemy zdefiniować w następujący sposób:

$$\min \sum_{i=1}^{n} \sum_{j=1}^{m} f_{ij}(x_{ij})$$

Przy spełnionych ograniczeniach:
$$\sum_{j=1}^{m} x_{ij} = supply(i), \text{ dla } i = 1, 2, \dots, n$$
$$\sum_{i=1}^{n} x_{ij} = demand(j), \text{ dla } j = 1, 2, \dots, m$$
$$x_{ij} \ge 0, \text{ dla } i = 1, 2, \dots, n \text{ i } j = 1, 2, \dots, m$$

Pierwszy zestaw ograniczeń mówi o tym, że całkowita ilość towaru transportowana z pojedynczego punktu nadania musi być równa jego początkowej 
ilości znajdującej się w tym punkcie. Z kolei drugi zestaw mówi o tym, że całkowita ilość towaru transportowana do pojedynczego punktu odbioru 
musi być równa jego początkowemu zapotrzebowaniu. W przypadku zadania niezbilansowanego równości w dwóch pierwszych zestawach ograniczeń, należy 
zmienić na odpowiednie nierówności. 

Zadanie jest liniowe, jeśli koszt transportu między punktami nadania i odbioru jest wprost proporcjonalny 
do ilości transportowanego towaru, tzn. jeśli $f_{ij}(x_{ij}) = c_{ij} x_{ij}$, gdzie $c_{ij}$ jest jednostkowym kosztem transportu 
między $i$-tym punktem nadania, a $j$-tym punktem odbioru. W przypadku kiedy zależność między kosztem transportu, a ilością transportowanego 
towaru wygląda inaczej, mówimy o zadaniu nieliniowym.

\subsection{Wersja liniowa}
Zadanie transportowe w wersji liniowej możemy przedstawić jako problem programowania liniowego. Optymalne rozwiązanie możemy więc wyznaczyć 
przy pomocy znanych metod używanych przy tej klasy problemach, takich jak np. metoda sympleks\cite{TP-SYMPLEX-BOOK}. 

Metoda ta polega na 
iteracyjnym wyszukiwaniu coraz lepszych rozwiązań zdefiniowanego problemu. Na początku wyznaczamy rozwiązanie początkowe, będące wierzchołkiem 
przestrzeni dopuszczalnych rozwiązań. Obliczamy dla niego wartość maksymalizowanej funkcji celu i następnie odrzucamy wszystkie 
wierzchołki, w których funkcja celu przyjmuje mniejsze wartości. W kolejnej iteracji przechodzimy do wierzchołka graniczącego z odnalezionym 
punktem, dla którego funkcja celu osiąga lepszą wartość i powtarzamy powyższe kroki. Algorytm kończy działanie w momencie, kiedy nie możemy 
znaleźć lepszego rozwiązania do kolejnej iteracji.

Zadanie transportowe posiada również interpretację sieciową. Dla $n$ punktów nadania i $m$ punktów odbioru możemy zdefiniować taki graf 
skierowany, który ma $n + m$ wierzchołków, gdzie $n$ wierzchołków odpowiada punktom nadania, oraz $m$ wierzchołków odpowiada punktom odbioru.
Wierzchołki są połączone krawędziami w taki sposób, że każdy z wierzchołków nadania posiada $m$ krawędzi, po jednej skierowanej do 
każdego z wierzchołków odbioru. Koszt na krawędzi jest kosztem jednostkowym przewozu towaru. Możemy je więc łatwo sprowadzić do zadania 
największego przepływu\cite{MAX-FLOW-ALG}. 

\subsection{Wersja nieliniowa}
O ile liniowa wersja zadania jest stosunkowo łatwa w rozwiązaniu, o tyle dla wersji nieliniowej nie ma ogólnej metody rozwiązywania. Należy 
ono do problemów z kategorii NP-trudnych\cite{Guisewite1990, NONLINEAR-NP-HARD}. Do jego rozwiązania używa się różnych algorytmów wyznaczających 
rozwiązania przybliżone, takich jak algorytmy metaheurystyczne. 

Wersja nieliniowa lepiej oddaje rzeczywisty problem planowania dostaw, gdzie koszty zależą od wielu czynników, które wpływają na to, 
że zależność między kosztem, a ilością transportowanego towaru nie jest liniowa. Niniejsza praca skupia się głównie na takich problemach.
