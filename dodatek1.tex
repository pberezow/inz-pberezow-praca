\chapter{Zawartość płyty CD}
\thispagestyle{chapterBeginStyle}
\label{plytaCD}

\section{Wymagania i instalacja}

Opisywany w poprzednich rozdziałach algorytm został zaimplementowany w języku Julia w wersji 1.3. Była testowana na komputerze z zainstalowanym 
systemem Ubuntu 18.04 w wersji, dla architektury 64-bitowej. Prezentowany pakiet powinien jednak działać na każdym systemie, na którym 
Julia 1.3 działa poprawnie.

Dodatkowo pakiet wymaga zainstalowanych następujących bibliotek:

\begin{itemize}
    \item Random
    \item JSON
    \item PyPlot
    \item Distributions
    \item JuMP(skrypt \textit{optimizers.jl})
\end{itemize}

W przypadku, jeśli wymagane biblioteki nie są zainstalowane należy otworzyć \textit{REPL} Julii i przy pomocy modułu \textit{Pkg} dodać wymagane 
biblioteki poprzez wywołanie funkcji \textit{Pkg.add()}, gdzie w argumencie podajemy nazwę instalowanego pakietu, np:

\begin{lstlisting}
    Pkg.add("Distributions")
\end{lstlisting}

Po zainstalowaniu wyżej wymienionych zależności, biblioteka jest gotowa do użytku. Instrukcja obsługi została zamieszczona w kolejnych sekcjach.

Udostępniony pakiet zawiera zestaw testów jednostkowych, które weryfikują poprawność działania składowych algorytmu. Dodatkowo załączono przykładowy 
plik konfiguracyjny oraz definicje kilku funkcji kosztu.

W katalogu głównym znajdują się 3 foldery:

\begin{itemize}
    \item \textbf{src} - zawiera implementacje opisywanego algorytmu
    \item \textbf{tests} - zawiera testy jednostkowe
    \item \textbf{examples} - zawiera przykładowe pliki konfiguracyjne
\end{itemize}

W katalogu \textbf{src} znajdują się pliki:

\begin{itemize}
    \item \textbf{chromosom.jl} - zawiera definicje struktury reprezentującej pojedynczego osobnika w populacji. Znajdują się tutaj też wszystkie 
        funkcje operujące na zdefiniowanej strukturze takie jak operatory mutacji, krzyżowania, inicjalizacja, walidacja rozwiązania.
    \item \textbf{population.jl} - zawiera definicje struktury reprezentującej całą populację rozwiązań. Znajdują się tutaj wszystkie główne 
        funkcje operujące na populacji.
    \item \textbf{config.jl} - zawiera definicje struktury pliku konfiguracyjnego. Znajdują się tutaj funkcje odpowiedzialne za zapisywanie i wczytywanie 
        pliku konfiguracyjnego.
    \item \textbf{functions.jl} - w tym pliku znajdują się definicje wszystkich dostępnych funkcji celu. Każda taka funkcja powinna przyjmować 
        jeden argument typu \textit{Array\{Float64, 2\}}. Zdefiniowano tutaj też słownik, który mapuje nazwy funkcji do ich implementacji. Aby dodać 
        nową funkcje celu, użytkownik powinien zaimplementować ją w tym pliku, a następnie przypisać ją do konkretnej nazwy w słowniku.
    \item \textbf{GeneticNTP.jl} - zawiera definicje całego modułu o nazwie \textit{GeneticNTP}.
\end{itemize}

Dodatkowo w katalogu głównym znajduje się pliki:
\begin{itemize}
    \item \textit{run\_example.jl} - uruchamia przykładowe zadanie transportowe. Komentarze w pliku opisują krok po kroku jak używać algorytmu 
        korzystając z \textit{REPL} Julii.
    \item \textit{run.jl} - umożliwia uruchomienie programu dla dowolnego pliku konfiguracyjnego za pomocą polecenia:
    \begin{lstlisting}[language=bash]
$ julia run.jl plik_konfiguracyjny ilość_pokoleń nazwa_funkcji_kosztu
    \end{lstlisting}
    \item \textit{optimizers.jl} - umożliwia optymalizacje zadania transportowego z użyciem solverów \textbf{GLPK, Ipopt} i \textbf{CPLEX}. 
        Używa biblioteki JuMP w celu uruchomienia sloverów.
    \item \textit{configToGams.jl} - umożliwia konwersje pliku konfiguracyjnego na plik wejściowy GAMS.
\end{itemize} 

Przykładowe uruchomienie skryptu \textit{run\_example.jl}:

[TODO: uruchomienie]

Przykładowe uruchomienie skryptu \textit{optimizers.jl}:

\begin{lstlisting}[language=bash, frame=single]
piotr@piotr-hp:~$ julia optimizers.jl ./examples/ex_7x7.json GLPK Linear
START
Config loaded.
Using GLPK...
DONE.
Status: OPTIMAL
Result: 1132.0
\end{lstlisting}

\section{Pliki konfiguracyjne}

Do biblioteki został dodany moduł obsługi plików konfiguracyjnych. Używają one formatu JSON. Moduł pozwala na zapisywanie i wczytywanie 
całej konfiguracji algorytmu, w skład której wchodzą wektory popytu i podaży, macierz kosztu, oraz wszystkie parametry programu.
Definicja poszczególnych parametrów:

\begin{itemize}
    \item $populationSize$ - rozmiar całkowitej populacji. Powinien być dodatnią liczbą całkowitą.
    \item $eliteProc$ - ułamek najlepszych rozwiązań, które zostają przepisane do następnego pokolenia. 
    Wartość powinna znajdować się w przedziale $[0, 1]$. Testy pokazują, że najlepsze rozwiązania są generowane dla wartości parametru w przedziale
    $[0.1, 0.3]$.
    \item $mutationProb$ - prawdopodobieństwo mutacji. Przyjmuje wartość z zakresu $[0, 1]$. Warto pamiętać o tym, że zalecane prawdopodobieństwo 
    mutacji nie powinno przekraczać kilkanastu procent.
    \item $mutationRate$ - wielkość mutacji, określa stosunek rozmiaru podmacierzy, wybieranej do ponownej inicjalizacji podczas mutacji, 
    do macierzy rozwiązania. Przyjmuje wartości z zakresu $[0, 1]$. Tak jak w przypadku prawdopodobieństwa mutacji, jej wielkość nie powinna 
    przekraczać kilkunastu procent, ponieważ zbyt duża mutacja może niszczyć znalezione rozwiązania.
    \item $crossoverProb$ - prawdopodobieństwo krzyżowania. Przyjmuje wartości z przedziału $[0, 1]$. Zaleca się ustawialnie wartości z przedziału 
    $[0.5, 0.9]$. Należy pamiętać o tym, że suma parametrów eliteProc i crossoverProb nie może być większa niż $1$.
    \item mode - tryb w jakim ma działać algorytm. Przyjmuje wartości $regular$(w przypadku wyboru klasycznego modelu) lub $island$(w przypadku 
    modelu wyspowego).
    \item $numberOfSeparateGenerations$ - liczba naturalna określająca ilość iteracji jakie wykona algorytm pomiędzy rozdzieleniem populacji na 
    mniejsze części, a ponownym jej scaleniem. Dla wyboru modelu klasycznego należy ustawić wartość parametru na $1$.
\end{itemize}

Dodatkowo w pliku konfiguracyjnym określamy wektor popytu, podaży i macierz kosztów:

\begin{itemize}
    \item $demand$ - wektor popytu. Przyjmuje jako wartość listę elementów, które składają się z dwóch pól - $i$, które określa indeks wektora i 
    $val$, które określa wartosć wektora w miejscu $i$ - $demand[i] = val$.
    \item $supply$ - wektor podaży. Przyjmuje jako wartość listę elementów, o polach takich samych jak w przypadku wektora popytu. Pojedynczy element 
    opisuje pole wektora $supply[i] = val$.
    \item $costMatrix$ - macierz kosztu, która może być wykorzystywana w funkcji celu. Przyjmuje jako wartość listę elementów, które składają się 
    z trzech pól: $s$ - określa indeks odpowiadający indeksowi wektora podaży, $d$ - określa indeks odpowiadający indeksowi wektora popytu, oraz 
    $val$ - określa wartość macierzy w polu o podanych indeksach $costMatrix[d, s] = val$.
\end{itemize}

Przykładowy plik konfiguracyjny:

\begin{lstlisting}[language=json, firstnumber=1, frame=single]
{
    "populationSize": 100,
    "eliteProc": 0.3,
    "mutationProb": 0.1,
    "mutationRate": 0.05,
    "crossoverProb": 0.2,
    "mode": "regular",
    "numberOfSeparateGenerations": 1,
    "demand": [
        {
            "val": 1.0,
            "i": 1
        },
        {
            "val": 10.0,
            "i": 2
        }
    ],
    "supply": [
        {
            "val": 1.0,
            "i": 1
        },
        {
            "val": 5.0,
            "i": 2
        },
        {
            "val": 5.0,
            "i": 3
        }
    ],
    "costMatrix": [
        {
            "val": 1.0,
            "s": 1,
            "d": 1
        },
        {
            "val": 2.0,
            "s": 2,
            "d": 1
        },
        {
            "val": 3.0,
            "s": 3,
            "d": 1
        },
        {
            "val": 10.0,
            "s": 1,
            "d": 2
        },
        {
            "val": 20.0,
            "s": 2,
            "d": 2
        },
        {
            "val": 30.0,
            "s": 3,
            "d": 2
        }
    ]
}
\end{lstlisting}

\section{Uruchomienie biblioteki}

Przed uruchomieniem biblioteki należy zdefiniować w pliku \textit{functions.jl} funkcję kosztu dla zadania transportowego, lub skorzystać z 
wcześniej zdefiniowanych funkcji znajdujących się już w tym pliku. Następnie musimy stworzyć plik konfiguracyjny, według opisu znajdującego 
się w poprzedniej sekcji.

W celu uruchomienia biblioteki możemy używać przygotowanego skryptu znajdującego się w pliku \textit{run.jl}. Uruchamiamy go poprzez wywołanie 
interpretera julii z argumentami:
\begin{enumerate}
    \item ścieżka do pliku \textit{run.jl}
    \item ścieżka do pliku konfiguracyjnego 
    \item maksymalna ilość pokoleń
    \item nazwa funkcji kosztu
\end{enumerate}

Przykład uruchomienia pliku:

\begin{lstlisting}[language=bash, frame=single]
piotr@piotr-hp:~/geneticNTP$ julia run.jl ./examples/ex_7x7.json 1000 B
Path: ./examples/ex_7x7.json
Running on 4 threads
==================== CONFIG ====================
Population Size: 100
Crossover: 70.0%
Mutation: 10.0% / rate: 0.05
Elite: 30.0%
Iterations: 1000
Cost Func: B

Best result in first generation: 660.8
Done in: 0.52914400100708 s
Best result: 265.8226422107641
\end{lstlisting}

W przypadku, jeśli chcielibyśmy używać biblioteki z poziomu \textit{REPL} Julii musimy wykonać następujące kroki:

\begin{enumerate}
    \item Włączamy \textit{REPL} Julii.
\begin{lstlisting}[language=bash, frame=single]
piotr@piotr-hp:~$ julia 
               _
   _       _ _(_)_     |  Documentation: https://docs.julialang.org
  (_)     | (_) (_)    |
   _ _   _| |_  __ _   |  Type "?" for help, "]?" for Pkg help.
  | | | | | | |/ _` |  |
  | | |_| | | | (_| |  |  Version 1.3.0 (2019-11-26)
 _/ |\__'_|_|_|\__'_|  |  Official https://julialang.org/ release
|__/                   |

julia> 

\end{lstlisting}

    \item Dołączamy bibliotekę \textit{GeneticNTP}.
\begin{lstlisting}[language=bash, frame=single]
julia> include("GeneticNTP.jl")
Main.GeneticNTP

julia> using .GeneticNTP
\end{lstlisting}

    \item Uruchamiamy algorytm wywołując funkcję \textit{runEA()}, gdzie w argumentach podajemy ścieżkę do pliku konfiguracyjnego, 
        maksymalną ilość pokoleń oraz nazwę funkcji kosztu z słownika znajdującego się w pliku \textit{functions.jl}
\begin{lstlisting}[language=bash, frame=single]
julia> c = GeneticNTP.runEA("../examples/ex_7x7.json", 1000, "A")
==================== CONFIG ====================
Population Size: 100
Crossover: 70.0%
Mutation: 10.0% / rate: 0.05
Elite: 30.0%
Iterations: 1000
Cost Func: A

Best result in first generation: 643.0    
\end{lstlisting}

    \item Wynikiem jest obiekt \textit{Chromosom}, który w polu \textit{result} posiada macierz rozwiązania, a w polu \textit{cost} koszt 
    znalezionego rozwiązania.
    
\begin{lstlisting}[language=bash, frame=single]
julia> c.result
7 x 7 Array{Float64,2}:
18.0516   0.386976  0.424021  0.285102  0.235168  0.273782  0.34338 
0.674536  16.1462   0.463446  0.999492  0.509193  0.40339   0.80372 
1.50289   1.54443   14.8906   0.809828  0.317118  0.306721  0.628409
1.90339   1.74331   3.83536   12.4525   0.0       1.31811   1.74737 
1.72222   1.47616   1.70419   3.91253   16.5287   0.391483  0.26473 
1.75765   1.10214   1.68994   0.975765  1.53068   16.6977   1.24612 
1.38773   5.60076   1.99244   0.564831  0.879151  0.608813  14.9663 

julia> c.cost
183.0

\end{lstlisting}

\end{enumerate}

