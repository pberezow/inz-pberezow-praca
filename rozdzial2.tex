\chapter{Równoległą implementacja algorytmu ewolucyjnego}
\thispagestyle{chapterBeginStyle}

\section{Użyte technologie}
Do implementacji algorytmu zastosowano język Julia\cite{JULIA-PUB} w wersji 1.3. Julia jest stosunkowo nowym językiem programowania. Został on zaprojektowany 
z myślą o zastosowaniach w obliczeniach numerycznych. Łączy on w sobie zalety języków niskopoziomowych i wysokopoziomowych takie jak szybkość 
i czytelność kodu. Testy pokazują, że program napisany w Julii może być równie szybki, jak program napisany w C\cite{JULIA-PERFORMANCE}. 

Ostatnie aktualizacje w znaczącym stopniu rozwinęły wsparcie języka dla obliczeń równoległych. W tym momencie Julia oferuje wsparcie dla 
równoległości na poziomie wątków jak i procesów, co dodatkowo wpłynęło na wybór tego właśnie języka. 


\section{Reprezentacja chromosomu}
W opisywanej implementacji jako reprezentacje rozwiązania przyjęto macierz:
$$V = (v_{ij}) \text{, gdzie } 1 \le i \le length(demand) \land 1 \le j \le length(supply)$$

Rozwiązanie jest zakodowane w taki sposób, że komórka macierzy o indeksie $[i, j]$ określa ilość transportowanego towaru 
między $i$-tym punketm nadania i $j$-tym punktem odbioru. Jest to jedna z najbardziej naturalnych reprezentacji rozwiązania 
dla zadania transportowego.

\begin{table}[h!]
    \begin{center}
        \begin{tabular}{c|ccccc}
            i\textbackslash{j} & 10.0 & 7.0 & 5.0 & 13.0 & 12.0 \\ 
            \hline
            12.0 & 0.0 & 7.0 & 5.0 & 0.0 & 0.0 \\
            10.0 & 5.0 & 0.0 & 0.0 & 0.0 & 5.0 \\
            3.0 & 3.0 & 0.0 & 0.0 & 0.0 & 0.0 \\
            10.0 & 0.0 & 0.0 & 0.0 & 3.0 & 7.0 \\
            12.0 & 2.0 & 0.0 & 0.0 & 10.0 & 0.0 \\
        \end{tabular}
    \end{center}
    \caption{Przykładowe rozwiązanie(pierwszy wiersz - wektor punktów odbioru, pierwsza kolumna - wektor punktów nadania).}
\end{table}

Aby ograniczenia zadania zostały zachowane macierz rozwiązania musi spełniać następujące warunki:
$$\sum_{j=1}^{m} v_{ij} = supply[i], \text{ dla } i = 1, 2, \dots, n \text{, gdzie } n = length(demand)$$
$$\sum_{i=1}^{n} v_{ij} = demand[j], \text{ dla } j = 1, 2, \dots, m \text{, gdzie } m = length(supply)$$
$$v_{ij} \ge 0, \text{ dla } i = 1, 2, \dots, n \text{ i } j = 1, 2, \dots, m$$


\section{Inicjalizacja chromosomu}
[TODO: Opis]
[TODO: Przykład]


\section{Operator krzyżowania}
Operator krzyżowania został zdefiniowany jako kombinacja wypukła dwóch rodziców.

\begin{pseudokod}
    \KwData{$P_1, P_2$ - rodzice wybrani do krzyżowania}
    \KwResult{$O_1, O_2$ - otrzymane dzieci}
    $c_1 \gets rand(0,\dots,1)$\;
    $c_2 \gets 1.0 - c_1$\;
    $O_1 \gets c_1 * P_1 + c_2 * P_2$\;
    $O_2 \gets c_2 * P_1 + c_1 * P_2$\;
    \Return{$(O_1, O_2)$}\;
\end{pseudokod}

Zdefiniowany tak operator krzyżowania nie narusza ograniczeń zadania, ponieważ przestrzeń rozwiązań jest wypukła. Wynika z tego, że jeśli 
rodzice spełniali ograniczenia, to otrzymane w ten sposób dzieci również muszą spełniać ograniczenia.

[TODO: Dodać przykład zastosowania operatora]

\section{Operator mutacji}
Operator mutacji opiera się na modyfikacji rozwiązania poprzez wybranie z niego podmacierzy i jej ponowną inicjalizacje. Załóżmy, że 
mamy $n$ punktów nadania i $m$ punktów odbioru. Wybierzmy jako kandydata do mutacji macierz $V = (v_{ij})$, gdzie $1 \le i \le n$ i 
$1 \le j \le m$. Podmacierz $W = w_{ij}$ jest tworzona w następujący sposób:

\begin{itemize}
    \item Losujemy podzbiór $k$ indeksów $\{i_1, \dots, i_k\}$ ze zbioru $\{1, \dots, n\}$ oraz podzbiór $l$ indeksów $\{j_1, \dots, j_l\}$ 
    ze zbioru $\{1, \dots, m\}$, $2 \le k \le n$ i $2 \le l \le m$.
    \item Tworzymy podmacierz $W$ składającą się z takich elementów macierzy $V$, które zostały wylosowane, tzn. element $v_{ij} \in V$ 
    zostaje włączony do podmacierzy $W$ tylko jeśli $i \in \{i_1, \dots, i_n\}$ oraz $j \in \{j_1, \dots, j_l\}$.
\end{itemize}

Dla stworzonej macierzy $W$ tworzymy nowe wektory $demand_W$ i $supply_W$ w następujący sposób:
$$demand_W[i] = \sum_{j \in \{j_1, \dots, j_l\}} v_{ij}, \text{ dla } 1 \le i \le k$$
$$supply_W[j] = \sum_{i \in \{i_1, \dots, i_k\}} v_{ij}, \text{ dla } 1 \le j \le l$$

Następnie na nowo inicjalizujemy podmacierz $W$ macierzy $V$ używając stworzonych wektorów $demand_W$ oraz $supply_W$ do określenia popytu i podaży
w wybranych punktach nadania i odbioru. Po zakończeniu inicjalizacji przepisujemy wartości z podmacierzy $W$ z powrotem do macierzy $V$.

\begin{pseudokod}
    \KwData{$V$ - osobnik wybrany do mutacji wielkości $n \times m$, $k, l$ - wielkość podmacierzy}
    \KwResult{$V$ - osobnik po mutacji}
    $demand\_idx \gets$ wylosuj podzbiór długości $k$ ze zbioru $\{1, \dots, n\}$\;
    $supply\_idx \gets$ wylosuj podzbiór długości $l$ ze zbioru $\{1, \dots, m\}$\;
    $W \gets zeros(k, l)$ \tcc*[r]{generujemy macierz zerową rozmiaru $k \times l$}
    \For{$i \in \{1, \dots, k\}$} {
        \For{$j \in \{1, \dots, l\}$} {
            $W[i, j] \gets V[demand\_idx[i], supply\_idx[j]]$\;
        }
    }
    $demand\_vec \gets zeros(k)$ \tcc*[r]{generujemy wektor zerowy długości k}
    $supply\_vec \gets zeros(l)$ \tcc*[r]{generujemy wektor zerowy długości l}
    \For{$i \in demand\_idx$} {
        $demand\_vec[i] \gets \sum_{j \in supply\_idx} V[i, j]$\;
    }
    \For{$j \in supply\_idx$} {
        $supply\_vec[j] \gets \sum_{i \in supply\_idx} V[i, j]$\;
    }
    $W \gets inicjalizacja(W, demand\_vec, supply\_vec)$\;
    \For{$i \in \{1, \dots, k\}$} {
        \For{$j \in \{1, \dots, l\}$} {
            $V[demand\_idx[i], supply\_idx[j]] \gets W[i, j]$\;
        }
    }
    \Return{$V$}
\end{pseudokod}

Zdefiniowano dwa operatory mutacji różniące się jedynie procedurą inicjalizacji. W pierwszym używamy tej samej procedury, którą inicjalizujemy 
nowe chromosomy podczas generowania populacji początkowej. Druga jest modyfikacją tej procedury. Modyfikacja polega na tym, że zamiast wybierać 
jako wartość pola $val = min(demand[i], supply[j])$ wybieramy liczbę z zakresu $[0, val]$. Zmiana ta powoduje, że otrzymana macierz może naruszać 
ograniczenia zadania, dlatego po wstępnym wyznaczeniu wartości naprawiamy rozwiązanie poprzez zrobienie wymaganych dodawań w taki sposób, żeby 
rozwiązanie spełniało ograniczenia.

[TODO: Pseudokod]
[TODO: Przykład]

\section{Funkcje oceny}


\section{Metoda selekcji}
W algorytmie zastosowano standardową metodę selekcji - metodę koła ruletki. W tej metodzie lepiej przystosowane osobniki mają odpowiednio większe 
szanse na to, że zostaną wybrane do puli rodziców dla następnej generacji. Na początku obliczana jest suma kumulatywna dla całego pokolenia. Następnie 
otrzymane wartości skalujemy tak, aby znajdowały się w przedziale $[0, 1]$. W ten sposób przedział $[0, 1]$ zostaje podzielony na mniejsze, które 
odpowiadają pojedynczym osobnikom z populacji. Następnie generujemy losową liczbę z tego przedziału. Do zbioru rodziców dodawany jest osobnik, 
w którego przedziale znajduje się wylosowana liczba. Warto zauważyć że tak zaprojektowana selekcja pozwala na to, że jeden osobnik zostanie wybany 
kilkukrotnie na rodzica.
[TODO: obrazek ilustrujący algorytm]

\section{Wersja równoległa}
Algorytm może działać w dwóch trybach:
\begin{itemize}
    \item Klasycznym
    \item Wyspowym
\end{itemize}

Model klasyczny nie różni się wiele od standardowego algorytmu ewolucyjnego. Mamy tutaj standardowo jedną populacje, która ewoluuje przez 
określoną na starcie liczbę pokoleń. Wszystkie dostępne parametry algorytmu oraz ich dopuszczalne wartości opisane są w dalczej części pracy, 
w sekcji \textit{Pliki konfiguracyjne}.

W modelu klasycznym zrównoleglenie odbywa się na poziomie pojedynczej iteracji. Ewolucje populacji możemy podzielić tutaj na dwie części:
\begin{itemize}
    \item Krzyżowanie - w tej części między wątki rozdzielani są rodzice wybrani do krzyżowania. Następnie każdy z wątków generuje swoją część 
    dzieci, następnie z określonym prawdopodobieństwem stostuje na dzieciach operator mutacji i ostatecznie oblicza dla nich wartość funkcji oceny. 
    Dzieci są dodawane do kolejnej populacji.
    \item Dopełnienie populacji - w tej części do kolejnej populacji przepisywana jest część najlepszych rozwiązań oraz losowo wybranych osobników 
    z poprzedniej populacji. Następnie te dodane osobniki są poddawane mutacji i obliczana jest dla nich funkcja oceny.
\end{itemize}

Model wyspowy różni się od klasycznego podejścia tym, że całkowita populacja jest tutaj rozdzielana na kilka mniejszych. Następnie każda z 
populacji częściowych ewoluuje niezależnie od innych przez określoną liczbę pokoleń. Po zakończeniu tego procesu wszystkie częściowe populacje 
są na nowo łączone w jedną. Następnie najlepsze rozwiązywanie jest zapisywane, a populacja zostaje na nowo podzielona na kilka mniejszych i 
cały proces się powtarza, do momentu w którym ilość generacji przekroczy określoną na początku liczbę. Na końcu najlepsze rozwiązanie jest zwracane.

W tym modelu zrównoleglenie obliczeń polega na tym, że przy każdym podziale populacji jeden wątek zarządza pojedynczą częścią populacji.
Model ten skaluje się lepiej niż model klasyczny ze względu na to że podzadania przydzielane wątkom są większe. Minusem jest tutaj to, że populacja 
musi być odpowiednio duża, żeby jej podział na mniejsze części się sprawdził.
[TODO: Algorytm w wersji graficznej]

W tym momencie każda z częściowych populacji w modelu wyspowym posiada takie same parametry. [...]

\section{Pliki konfiguracyjne}
Dodatkowo do algorytmu został dodany moduł obsługi plików konfiguracyjnych. Używają one formatu JSON. Moduł pozwala na zapisywanie i wczytywanie 
całej konfiguracji algorytmu, w skład której wchodzą wektory popytu i podaży, macierz kosztu, oraz wszystkie parametry programu.
Definicja poszczególnych parametrów:

\begin{itemize}
    \item populationSize - rozmiar całkowitej populacji. Powinien być dodatnią liczbą całkowitą.
    \item eliteProc - ułamek najlepszych rozwiązań, które zostają przepisane do następnego pokolenia. 
    Wartość powinna znajdować się w przedziale $[0, 1]$. Testy pokazują, że najlepsze rozwiązania są generowane dla wartości parametru w przedziale
    $[0.1, 0.3]$.
    \item mutationProb - prawdopodobieństwo mutacji. Przyjmuje wartość z zakresu $[0, 1]$. Warto pamiętać o tym, że zalecane prawdopodobieństwo 
    mutacji nie powinno przekraczać kilkanastu procent.
    \item mutationRate - wielkość mutacji, określa stosunek rozmiaru podmacierzy, wybieranej do ponownej inicjalizacji podczas mutacji, 
    do macierzy rozwiązania. Przyjmuje wartości z zakresu $[0, 1]$. Tak jak w przypadku prawdopodobieństwa mutacji, jej wielkość nie powinna 
    przekraczać kilkunastu procent, ponieważ zbyt duża mutacja może niszczyć znalezione rozwiązania.
    \item crossoverProb - prawdopodobieństwo krzyżowania. Przyjmuje wartości z przedziału $[0, 1]$. Zaleca się ustawialnie wartości z przedziału 
    $[0.5, 0.9]$. Należy pamiętać o tym, że suma parametrów eliteProc i crossoverProb nie może być większa niż $1$.
    \item mode - tryb w jakim ma działać algorytm. Przyjmuje wartości "regular"(w przypadku wyboru klasycznego modelu) lub "island"(w przypadku 
    modelu wyspowego).
    \item numberOfSeparateGenerations - liczba naturalna określająca ilość iteracji jakie wykona algorytm pomiędzy rozdzieleniem populacji na 
    mniejsze części, a ponownym jej scaleniem. Dla wyboru modelu klasycznego należy ustawić wartość parametru na $1$.
\end{itemize}

Dodatkowo w pliku konfiguracyjnym określamy wektor popytu, podaży i macierz kosztów:

\begin{itemize}
    \item demand - wektor popytu. Przyjmuje jako wartość listę elementów, które składają się z dwóch pól - "i", które określa indeks wektora i 
    "val", które określa wartosć wektora w miejscu $i$ - $demand[i] = val$.
    \item supply - wektor podaży. Przyjmuje jako wartość listę elementów, o polach takich samych jak w przypadku wektora popytu. Pojedynczy element 
    opisuje pole wektora $supply[i] = val$.
    \item costMatrix - macierz kosztu, która może być wykorzystywana w funkcji celu. Przyjmuje jako wartość listę elementów, które składają się 
    z trzech pól: "s" - określa indeks odpowiadający indeksowi wektora podaży, "d" - określa indeks odpowiadający indeksowi wektora popytu, oraz 
    "val" - określa wartość macierzy w polu o podanych indeksach $costMatrix[d, s] = val$.
\end{itemize}

Przykładowy plik konfiguracyjny:

[TODO: dodać przykładowy plik]