\chapter{Wstęp}
\thispagestyle{chapterBeginStyle}

Tematem niniejszej pracy jest ewolucyjne podejście do rozwiązywania zadania transportowego. Przedstawiono w niej elementy, z których składa się algorytm 
ewolucyjny, oraz zasadę jego działania. Następnie pokazano w jaki sposób można dostosować algorytm pod konkretny problem jakim 
jest omawiane zadanie transportowe. W tym celu zastosowano niestandardową reprezentację rozwiązania, oraz dostosowano do niej operatory mutacji 
i krzyżowania w taki sposób, żeby ich użycie nie naruszało ograniczeń zadania. W celu uzyskania lepszych wyników czasowych zaproponowano dwa 
modele zrównoleglenia algorytmu. 

Praca została podzielona na dwie części. Pierwsza z nich składa się z rodziałów ,,Analiza problemu'' oraz 
,,Ewolucyjne podejście do nieliniowego zadania transportowego''. Przedstawiono w niej problem transportowy i w sposób teoretyczny opisano budowę 
prezentowanego algorytmu. 
Znajduje się tutaj też opis użytych rozwiązań wraz z pseudokodem i przykładami, które w przejrzysty sposób pokazują zasadę ich działania. 
W drugiej części, w rozdziale zatytułowanym ,,Wyniki eksperymentalne'' przeprowadzono analizę eksperymentalną algorytmu ewolucyjnego, w której między 
innymi porównano go z dostępnymi solverami służącymi do optymalizacji. Pokazano również jaki zysk daje wprowadzenie opisanych modeli zrównoleglenia. 

Do implementacji opisanego rozwiązania użyto języka Julia w wersji 1.3. Szczegóły dotyczące implementacji znajdują się w dodatku \ref{plytaCD}. 
Wykresy przedstawione w części eksperymentalnej zostały stworzone przy użyciu bibliotyki \textit{PyPlot}. Dostęp do solverów był możliwy dzięki 
serwisowi NEOS, który został opsiany na początku rozdziału \ref{rozdzial3}.